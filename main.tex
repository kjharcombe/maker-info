\documentclass{article}
\usepackage[a4paper, margin=0.75in]{geometry}
\usepackage{tikz}
\usepackage{fontspec}
\usepackage{circuitikz}
\usepackage{url}
\title{The Hacker's Book of Reference Material}
\author{Various Authors}
\date{}
\begin{document}
\setmainfont{Gentium Book Basic}
\centering
The Maker's Bill of Rights
\begin{itemize}
    \item Meaningful and specific parts lists shall be included.
    \item Cases shall be easy to open.
    \item Special tools are allowed only for darn good reasons
    \item Profiting by selling expensive special tools is wrong and not making special tools available is even worse.
    \item Torx is OK; tamperproof is rarely OK.
    \item Components, not entire sub-assemblies, shall be replaceable.
    \item Consumables, like fuses and filters, shall be easy to access.
    \item Power from USB is good; power from proprietary power adapters is bad.
    \item Standard connectors shall have pinouts defined.
    \item If it snaps shut, it shall snap open
    \item Screws better than glues.
    \item Docs and drivers shall have permalinks and shall reside for all perpetuity at archive.org.
    \item Ease of repair shall be a design ideal, not an afterthought.
    \item Metric or standard, not both.
    \item Schematics shall be included.
\end{itemize}
\end{document}